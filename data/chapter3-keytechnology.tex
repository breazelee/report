% !TEX root = ../midtermreport.tex
% !Mode:: "TeX:UTF-8"
\chapter{关键技术和难点}
\label{chap:keytechnology}


\section{PBF流体仿真在云仿真中的运用}
\label{sec:fulid-simulation}
在云的建模和绘制中,基于流体仿真的云动态仿真方法一直是计算机图形学研究的难点,
同时在本课题中,流体的仿真过程使用PBF算法。


传统的流体仿真算法主要分为两类,一类是基于有网格的欧拉方法模拟流体运动,
一类是基于无网格的拉格朗日方法模拟流体运动。
其中网格方法是计算力学中求解工程问题的主要数值计算方法,
在求解流体运动,拍击等设计到特大变形的问题时,有限元网格可能会产生严重地扭曲,
这样不仅需要网格重构,而且严重地影响计算精度,对于流体运动、拍击、高速撞击等动态问题,
显式时间积分的步长取决于有限元网格的最小尺寸,因而网格的扭曲将使时间积分步长过长,
大幅度地增加了计算工作量;对裂纹的动态扩展问题,由于裂纹的扩展方向不能事先确定,
因而在计算过程中需要不断地重新划分网格以模拟裂纹的动态扩展过程;
有限元近似基于网格,因此必然难于处理与原始网格线不一致的不连续性和大变形;
复杂三维结构的有限元网格生成也是极具有挑战性的问题。

对于无网格法的研究可以追溯到20世纪70年代对非规则网格有限差分法的研究,
由于当时有限元法的巨大成功,这类方法没有受到重视。
\citeauthor{Lucy1977A}\upcite{Lucy1977A}和\citeauthor{Gingold1977Smoothed}
\upcite{Gingold1977Smoothed}等分别提出了光滑质点流体动力学(SPH)方法,
并且成功地应用于天体物理领域。Johnson等提出了归一化光滑函数算法,使其通过分片试验,
可以正确模拟常应变状态,提高了SPH的精度;Vignjevic等提出了克服零能量模态的具体方案;
Monaghan对SPH法进行了总结;
SPH法已经被应用于冲击波模拟、流体动力学、水下爆炸仿真模拟、
高速碰撞等材料动态响应的数值模拟等等领域。

但是由于SPH在计算过程中对步长的要求比较苛刻,因此Miles和Matthias\upcite{Macklin2013Position}在2013年提出了一种融入PBD方法的PBF算法,通过求解一组位置约束,从而强制保证密度不变。
此方法可以实现类似SPH的仿真过程,同时保留了几何关系的稳定性,
最重要的是保持满足实时应用中对长时间步长的要求。
通过引入一个人工压力用来改善粒子的分布,创建表面张力并且降低传统SPH对邻居的需求。
最后通过应用涡旋约束作为速度处理来解决能量损失的问题。

其算法详见算法\ref{alg:pbf-algorithm}。

\subsection{不可压缩性}
为了保持密度不变,本论文通过求解一个每一个非线性的约束系统,其中每一个粒子一个约束方程。
每一个约束方程都是一个有关粒子$P_i$的坐标以及该粒子邻居坐标的约束方程。根据Bodin等【】提出的,
在第$i$个粒子的密度的约束被定义为状态方程:
\begin{equation}
\label{equ:density_constraint}
C_i(P_i,\dots,P_n) = \frac{\rho_i}{\rho_0} - 1
\end{equation}
其中$\rho_0$是密度常量,这里指的是空气密度。而$\rho_i$是被标准SPH密度估计的值:
\begin{equation}
\label{equ:sph_density}
\rho_i=\sum_jm_jW(P_i-P_j,h)
\end{equation}
本文假设所有的粒子质量相同,并且、、、、从后面的公式能够看出来。
本论文在实现过程中使用的是Poly6\upcite{Ller2003Particle}核函数去估计粒子的密度,
使用Spiky\upcite{Ller2003Particle}核函数来计算梯度。


\subsection{拉伸不稳定性}
在SPH仿真中一个常遇到的问题就是由于粒子周边只有很少的邻居粒子时导致负压强,
不能位置稳定而产生粒子聚集的现象,如图【】。
可以通过clamping perssures去保持压力的非负性,但会减少粒子的凝聚性。
Clavet等【】在2005年使用一个二度临界压力项,
而Alduan and Otaduy【】使用离散元(discrete element,简称DEM)力推动粒子分离到平缓内核宽度的一半。
Schechter and Bridson【】在自由表面周边放置一些ghost particles,以保证近似的密度稳定

本文使用Monaghan【】提出的方法,在光滑核本身增加一个人工的压力:\ref{equ:tensile_instability}
\begin{equation}
\label{equ:tensile_instability}
s_{coor}=-k\left(\frac{W(P_i-P_j,h)}{W(\Delta{}q,h)}\right)^n
\end{equation}
其中$\Delta{}q$是一个有固定距离,且在光滑核半径内部的点,$k$是一个小的正常量。
当$|\Delta{}q|=0.1h\dots{}0.3h,k=0.1$且$n=4$能够产生比较好的效果。
本论文增加该项在粒子位置校正过程,如公式\ref{}所示。
\begin{equation}
\label{equ:update_position}
\Delta{}P_i=\frac{1}{\rho_0}\sum_j(\lambda_i+\lambda_j+s_{coor})\bigtriangledown{}W(P_i-P_j,h)
\end{equation}

这种纯粹的排斥力使粒子密度略低于rest的密度。
因此粒子将其邻居粒子拉向自己,就像类似Clavet【】等描述的表面张力的效果。
本文注意到这种效应是一个非物质的人工抗聚类方法,需要权衡聚集误差和表面张力强度。

没有了聚集问题,本算法不需要满足SPH对必须保持30-40个粒子的经验规则的约束,从而加快了计算。

\subsection{漩涡约束和粘度}
基于位置的方法会引入附加的多不期望的阻尼。Fedkiw等【】引入涡流约束到计算机图形学中
在解决烟的仿真过程中的数值计算上的时间花销。
这后来被Lentine等【upcite】扩展到\textbf{能源保护流体仿真}
在泡沫存活期间,hong等【】展现如何将涡流约束从网格角度转换为SPH的粒子并应用到断流仿真中。
-----------------
本论文使用涡流约束去代替损失的能量,如图\ref{}所示。
这要求首先计算涡流在粒子的位置,
为了能够使用Monaghan【】提出的估计量。
---------------
\begin{equation}
\label{equ:vorticity}
\omega_i=\Delta \times \textbf{v} = \sum_j\textbf{v}_{ij}\times\Delta_{P_j}W(P_i-P_j,h)
\end{equation}

其中$\textbf{v}_{ij}=\textbf{v}_j-\textbf{v}_i$。
一定应用了涡旋约束,使用位置矢量$\textbf{N}=\frac{\eta}{|\eta|}$
计算校正的力
\begin{equation}
\label{equ:vorticity}
\textbf{f}_i^{vorticity}=\varepsilon(\textbf{N}\times\omega_i)
\end{equation}
本论文并未使用Hong【】提出的会任意增加涡流的正规化的$\omega$,
相反我们使用Fedkiw等【】提出的未归一化的值,因为这样只会增加本身存在的涡流。

 此外本文应用Schechter and Bridson【】提出的对连贯运动模拟较好的XSPH粘度。
在我们的仿真中参数$c$的值为$0.01$。
\begin{equation}
\label{equ:xsph_viscosity}
v_i^{new}=v_i + c\sum_j\textbf{v}_{ij}\cdot{}W(P_i-P_j,h)
\end{equation}


\subsection{PBD算法理论}
现在插入一节介绍PBD(Position Based Dynamics)方法的基本背景,并展现如何体现粒子的不可压缩性。
PBD的目标是找寻满足等式\ref{}的粒子校正位置$\Delta{}P$。
\begin{equation}
\label{equ:deltaP_constraint}
C(P+\Delta{}P)=0
\end{equation}

通过求解沿着约束的梯度求解的一系列牛顿【】:
\begin{equation}
\label{equ:newton_1}
\Delta{}P \approx \Delta{}C(P)\lambda 
\end{equation}
\begin{IEEEeqnarray}{rCl}
C(P+\Delta{}P) & \approx & C(P) + \Delta{}C^T\Delta{}P = 0 \label{equ:newton_2} \\
& \approx & C(P) + \Delta{}C^T\Delta{}C\lambda = 0  \label{equ:newton_3}
\end{IEEEeqnarray}

------------Monaghan【】解释SPH的原因是函数的梯度 被定义为粒子的-------------
基于此原理,粒子$k$的约束函数\ref{equ:density_constraint}的梯度被表示为:
\begin{equation}
\label{equ:density_constraint_constraint}
\Delta_{P_k}C_i = \frac{1}{\rho_0}\sum_j\Delta_{P_k}W(P_i,P_j,h)
\end{equation}

根据粒子$k$是不是邻居粒子而有两种形式,
如公式\ref{equ:density_constraint_constraint_two_case}所示。
\begin{equation}
\label{equ:density_constraint_constraint_two_case}
\Delta_{P_k}C_i = \frac{1}{\rho_0}\left\{
\begin{array}{ll}
\sum_j\Delta_{P_k}W(P_i,P_j,h)  & \text{if } k = i \\
{-}\Delta_{P_k}W(P_i,P_j,h)  & \text{if } k = j
\end{array} \right.
\end{equation}

将公式 \ref{equ:density_constraint_constraint_two_case} 带入
公式 \ref{equ:newton_3} 可求解$\lambda$:
\begin{equation}
\label{equ:lambda}
\lambda_i = {-}\frac{C_i(P_i,\dots,P_n)}{\sum_k|\Delta_{p_k}C_i|^2}
\end{equation}
可以发现,对在约束中的所有的粒子都具有相同的$\lambda$值。

由于约束函数\ref{equ:density_constraint}是非线性的,且在光滑核半径边界的梯度是趋于零。
当粒子接近分离的时候公式 \ref{equ:lambda}的分母也趋于不稳定。
此问题在PCISPH算法中被解决,通过基于被填满邻居相关粒子结构预计算一个保守校正范围。

或者使用CFM(Constraint Force Mixing)【】调整约束。CMF的思路是通过混合一些在约束函数中反馈的约束力来缓解约束,
在PBD仿真的情况下此基础上,公式\ref{equ:newton_3}转换为
\begin{equation}
\label{equ:constrain_with_cfm}
C(P+\Delta{}P)   \approx  C(P) + \Delta{}C^T\Delta{}C\lambda  + \varepsilon\lambda= 0
\end{equation}
其中 $\varepsilon$ 是用户用来约束仿真过程而指定的松弛参数。此时$\lambda_i$ 表示为
\begin{equation}
\label{equ:lambda_with_cfm}
\lambda_i = {-}\frac{C_i(P_i,\dots,P_n)}{\sum_k|\Delta_{p_k}C_i|^2 + \varepsilon}
\end{equation}
并且总体的位置的校正$\Delta{}P_i$(包括在从邻居粒子密度约束)为:
\begin{equation}
\label{equ:total_position_updade}
\Delta{}P_i = \frac{1}{\rho_0}\sum_j(\lambda_i+\lambda_j)\Delta{}W(P_i-P_j,h)
\end{equation}
。














\section{时序数据校准}
\label{sec:time_var_update}
通常采用的时序插值算法大都是之间对两个或者多个时间序列的数据进行插值,
其中的插值算法有很多,然而本文是基于流体的仿真,
如果只是基于数学方法的插值,流体仿真的意义就没有了。
而完全依赖流体仿真的方法是无法实现时序的平滑插值。




























