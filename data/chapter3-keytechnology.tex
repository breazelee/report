% !TEX root = ../midtermreport.tex
% !Mode:: "TeX:UTF-8"
\chapter{关键技术和难点}

\section{光滑粒子流体仿真}

\subsection {光滑粒子流体动力学综述}
SPH方法是配点法的一种。
光滑粒子流体动力学(SPH)技术是有\citeauthor{Lucy1977A}\upcite{Lucy1977A}
和\citeauthor{Gingold1977Smoothed}\upcite{Gingold1977Smoothed}于1977年最早提出,
也是公认的最早出现真正意义的无网格方法。

这种方法最初的目的是用来模拟三维无界空间中天体的演化,
这是因为群体粒子的运动同液体或气体的流动很类似。
虽然有FDM和FEM那样的传统网格数值方法存在,
但是如前面所讨论的那样,网格方法在处理一些特殊问题时有困难,
这就激发研究人员寻找替换的办法去解决这些问题,因此SPH方法也就随之诞生。

不像在六十年代早期发表的著名的PIC方法,SPH方法不需要网格来计算空间导数,
这些粒子可以在空间内运动,携带有关的计算信息,
这样为解描述连续流体动力学守恒的偏微分方程形成了计算框架。

SPH方法最初发明的目的是为寻找一种模拟质点运动的无网格的、自适应的、拉格朗日方法。
如今,SPH方法正在很多领域被应用,应用得最广的还是在天体物理学领域,
例如
两个星体碰撞的模拟,
超新星、银河系的形成和崩溃,
中子星形成黑洞,
白矮星的单体或多体爆炸,
甚至宇宙的演化。
SPH方法还在计算流体力学或计算固体力学中也有广泛的运用,
例如
自由表面流动,
磁流体动力学,
多项流,
近似不可压缩流动,
重力波,
穿过多孔介质的流动,
热传导,
冲击模拟,
热变换和质量流,
冰和粘性结晶,
还有易啤固体的断裂模拟等等。
SPH方法在模拟大变形和冲击载荷方面最具优势,
一个重要的应用领域就是高速冲击问题(HVD)。
在HVI中,冲击波穿过固体而传播, 这种特性跟流体有几分类似,
Libersky和其合作者在SPH方法在冲击问题上的应用有突出的贡献,
SPH方法另外一个重要的应用是爆炸问题。
Swegte和 Attaway使用SPH方法进行了水下爆炸模拟的可行性研究。
刘桂荣和其合作者已经把SPH方法应用于模拟水的冲击减震方面的研究。

SPH方法的广泛应用促进了原始SPH方法的提高,在数值算法上得到了长足的进步。
对于SPH方法的一些固有缺点,一些改进方法也被提出。
Swegle等论证了拉力的不稳定性问题,这对结构强度有重要的意义。
Morris讨论了由于粒子的不相关性而导致最后解的不精确的问题。
在过去的几年中,
不同的修改和改正方法已经在修补相关性和提高SPH方法的精确性上进行了大量的尝试。
这些修改导致了SPH方法表达公式的多样性,
例如
Monaghan提出更有效率的对称公式,
\citeauthor{Johnson1996NORMALIZED}\upcite{Johnson1996NORMALIZED}
给出轴对称正交公式,
这样由于速度场的正交速度应力为定值,切向速度应力就能被准确地给出。
WK.Liu提出先求出核函数的粒子方法在粒子近似上能得到更加准确的结果。
Chen等提出改进光滑粒子方法,在问题域内和边界区域上都可以提高模拟的精度。
Randies和Libersky扩展了应力点方法,
促进了在多维空间中张力的不稳定性问题和零能量方法问题的发展, 
其他一些著名的SPH的改进方法还包括动最小二乘法求解粒子流体动力学,
由Bonet和Kulasegaram提出了积分核的改进。
如今SPH方法己经成为一种能够模拟普通流体动力学问题的方法。
虽然SPH方法己经在不同的领域有广泛的应用,但是依然有存在一些问题需
要进行更深入的研究。

\chapter{SPH基本公式与概念}

\qquad{}作为一种无网格方法,光滑质点流体动力和其它的粒子方法相比具有相似性,
同时也有它的特殊性。本章详细地介绍SPH方法的基本公式和概念。
SPH的近似方法是SPH方法的重要部分,该方法包括函数的积分描述,
函数以及导数的离散表达形式,同时给出推导复杂偏微分方程所使用的一些技巧。
最后对SPH方法中的一些关键问题,
例如
光滑长度、
人工粘性、
边界处理
和时间积分步长的选取
进行重点说明。
由于这些基本概念、必要公式和有关技巧对改进SPH方法,
理解SPH的各种变形形式有重要的作用,因此有必要对以上问题进行详细的讨论。

\section{SPH的基本概念}
正如第一章所介绍地那样,SPH方法应用于流体力学问题,
就要求解流体力学问题中关于密度、速度、动量、能量等区域变量的偏微分方程。
除了一些特殊情况,要得到这些的偏微分方程的解析解是不可能的。
因此,学者们正通过不同方法来得到数值解。
数值解法的过程大致相同:
首先,需对偏微分方程所涉及的变量区域进行离散,
然后得到能够求解任意一点处的区域函数和其导数的离散形式的偏微分方程,
最后求解这些偏微分方程。

为了实现以上的求解思路,SPH方法需要采取如下的办法:

\begin{enumerate}
\item 研究区域由一群任意分布的粒子来描述,这决定了SPH方法的无网格特性。
不使用网格系统来描述的数值方法并不难,
但是关键问题是如何保证数值解的稳定性,
特别是粒子在支持域内高度无序时。

\item 由于区域内的粒子没有网格的约束,因此这些粒子之间没有相关性。
这为数学上提供了SPH方法稳定性基础,
因为用积分形式描述粒子通常很稳定,只要数值积分能被正确地执行。

\item 函数$f(x)$采用$f(x)=\int\limits_\Omega{}f(x')W(x-x',h)dx'$的积分近似表达,
这种方法在SPH方法中就称为核估计(kemel estimation)。
核估计是一种使用粒子来做近似估计的方法,也称作核近似。
这种方法在区域变量和其导数的求解是通过支持域内相邻粒子的相关变量加权求和来实现的,
这样会产生带状或稀疏离散的系统矩阵,这种矩阵形式与计算效率紧密相关。
因为大变形问题需要大量的粒子来描述问题区域。
如果矩阵是满矩阵的话,这将花掉难以接受的时间来解如此大的系统方程。

\item 在每个时间步都要进行粒子近似,
因此相关变量值的确定取决于当前时间步时周围粒子的分布。
SPH的适应性是由每个时间步上在支持区域里粒子的分布来决定的,
而且SPH的自适应性在区域变量近似的前一步执行的。
为保证积分的准确性和数值的稳定性,在求和过程中需要使用足够多的粒子。

\item 进行粒子近似要把所有的以偏微分方程表达的区域函数
转换成只与时间有关的离散形式的普通微分方程。
普通的微分方程可以用显式的积分算法获得更快的步进方法,
这样就可以得到所有粒子所携带信息的时间历程,
这也是解决动力学问题的一种常用方法。
在SPH方法中,需要找到一种合适的方法来决定满足稳定积分所需的时间步长。

\end{enumerate}\qquad{}
以上五点说明SPH方法是一种无网格、自适应、稳定的Lagrangian解法。

\section{SPH方法的必要公式}
\subsection{函数的积分描述}
SPH方法有两个主要的步骤:
第一步是积分表达或区域函数的核近似;第二个步骤就是粒子近似。
在第一个步骤中,当核函数与一个任意函数相乘,
然后在支持域内积分就给出了一个函数核函数的积分表达形式。
这种函数的近似方法和更多细节参考Liu的专著\upcite{Lucy1977A}。
离散系统中某个粒子处函数的粒子近似是由支持域内粒子的相关值的和来实现的。

在SPH方法中,函数$f(x)$的积分表达的概念由恒等式\ref{sph-base-function}:

\begin{equation}
\label{sph-base-function}
f(x) = \int\limits_\Omega{}f(x')\delta(x-x')dx'
\end{equation}

$f(x)$是三维向量$x$的函数,$\delta(x-x')$是$\delta$函数

\begin{equation}
\label{delta-function}
\delta(x-x') = \left\{
\begin{array}{rl}
1 & x = x'\\
0 & x \ne x' 
\end{array} \right.
\end{equation}

在\ref{delta-function}式中,$\Omega$为包括向量$x$的积分体积,
\ref{sph-base-function}式表明一个函数可以用一种积分形式来表达,
由于使用$\delta$函数,\ref{sph-base-function}式的积分表达是精确成立的,
只要$f(x)$在$\Omega$中有意义并且连续。

如果$\delta(x-x')$函数被函数$W(x-x',h)$代替,$f(x)$的积分可表达成:
\begin{equation}
\label{sph-base-kernel}
f(x)=\int\limits_\Omega{}f(x')W(x-x',h)dx'
\end{equation}

$W(x-x',h)$为光滑核函数或核函数。核函数中,$h$为光滑长度,
它决定光滑函数的影响区域的半径。注意到只要$W(x-x',h)$不是占函数,
\ref{sph-base-kernel}分表达式就只是近似成立。

SPH方法中有个规矩,核近似算子由角括号标出,因此公式\ref{sph-base-kernel}写成:
\begin{equation}
\label{sph-base-kernel:s}
< f(x) >=\int\limits_\Omega{}f(x')W(x-x',h)dx'
\end{equation}

核函数$W(x-x',h)$的构造方法将在第三章详细探讨,这里只简单介绍所需的条件。
第一个条件称为核函数的正交条件:
\begin{equation}
\label{kernel-function-condition1}
\int\limits_\Omega{}W(x-x',h)dx = 1
\end{equation}
这个条件也称为一致性条件。因为核函数的积分值都具有一致性。

第二个条件规定核函数具有$\delta$函数的特性,当光滑长度$h$接近零时,得到
\begin{equation}
\label{kernel-function-condition2}
\lim_{h \rightarrow 0}W(x-x',h) = \delta(x-x')
\end{equation}

第三个条件称之为核函数的紧支条件(compact condition)
\begin{equation}
\label{kernel-function-condition3}
W(x-x',h) = 0 \qquad \text{当}|x-x'| > kh
\end{equation}



















