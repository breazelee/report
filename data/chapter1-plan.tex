% !TEX root = ../midtermreport.tex
% !Mode:: "TeX:UTF-8"

\chapter{论文工作计划}
本部分主要包括
\section{论文研究目标}

本论文研究的目标是针对热带气旋这一种复杂的天气系统,设计并实现一个热带气旋的动态仿真模拟算法。 
该算法使用气象数据(Weather Research and Forecasting Model,简称WRF)做初始状态,
同时以长时间跨度的时变气象数据为输入,
通过基于传输径向基函数(Advected Radial Basis Functions,ARBF)算法改进算法,
融合基于位置的流体仿真(position based Fluids,PBF)方法,
利用GPU加速计算云的仿真,输出指定时间跨度的粒子数据,在保留更多细节的同时,保证仿真过度自然流畅。

在此算法的基础上,结合基于粒子的云建模方法和基于粒子系统的LOD层级结构的绘制方法,
形成包括建模、仿真、绘制的热带气旋动态仿真系统。

\section{论文主要研究内容}

主要演技的内容包括以几个方面:

\subsection{云基于WRF数据的建模算法}

云的建模方法包括:

1、基于网格(拉格朗日)的建模方法、

2、基于粒子(欧拉)的建模方法,

3、基于纹理的非物理方法。

\subsection{基于多散射模型的云绘制算法}
云的绘制方法包括:

1、基于参与介质的光照计算模型

其中包括单散射模型和多散射模型

2、基于纹理的非物理绘制方法。

\subsection{基于位置的流体仿真算法}

流涕仿真算法包括

1、基于网格(拉格朗日)的模拟算法、

2、基于粒子(欧拉)的模拟算法,

3、基于纹理的非物理模拟算法。

本文为了基本上满足气象观测使用,因此全部选择基于欧拉离散物理方法的建模方法,
即选择使用基于粒子的云建模方法;
使用基于多散射模型的云绘制方法;
选择基于例子的模拟算法。


