% !TEX root = ../midtermreport.tex
% !Mode:: "TeX:UTF-8"

\chapter{论文工作计划}
\label{chap:research_plan}

\section{论文研究目标}
\label{sect:research_goal}

% 介绍热带气旋
本论文主要研究一种常见的天气现象---热带气旋的模拟与仿真技术。
热带气旋(Tropical Cyclone,简称TC)
指的是一种快速旋转的风暴天气系统,有一个低气压中心和螺旋结构的云体,
同时伴随着强风、雷电、暴雨等现象,也被称作台风、飓风、热带风暴、气旋风暴、热带低压等。
一个成熟的热带气旋通常有完整的结构,其结构如图\ref{fig:tc_structure}所示,主要包括风眼,地面低压,暖心,中心密集云层区,风眼墙,螺旋雨带,外散环流等构成。
\begin{figure}[h!]
	\centering
	\includegraphics[width=400bp]{figure/tcstructure.jpg}
	\caption{热带气旋结构图}
	\label{fig:tc_structure}
\end{figure}

% 介绍研究意义
热带气旋的模拟和仿真技术一直是气象、海洋等领域演技研究的重点和难点,
而在计算机图形图像学领域,由于其独特的结构,热带气旋也是作为一个重要的研究内容。
其中主要包括研究热带气旋的可视化效果,云的真实感绘制技术,以及热带气旋的模拟技术。
同时能够较真实的模拟热带气旋的运动以及真实的展现其结构,
对气象研究、海洋研究、气象预测等具有极其重要的意义。

% 研究现状
在计算机图形图形领域,对热带气旋方面的的研究主要集中在对云的建模和仿真。
随着虚拟场景渲染技术的发展,云的模拟表现也成为计算机图形学和虚拟现实技术中的热门课题。
云的真实感模拟不仅能有效提高场景逼真度,更能传达丰富的气象信息。
云作为一种常见的自然现象,由于其形状千变万化,形成、发展和消散的过程又极其复杂,
且具有水汽粒子的半透明特征。
能够实现即满足气象学应用,同时实现逼真展现的云场景模拟是一项很艰巨的任务。

早期在图形图像学领域云建模大体分为两类,
即基于过程的云建模和基于物理的云建模。
前者侧重于利用噪声、纹理或者交互式的手段对云进行建模,通常需要经历繁琐的参数调整;
后者则通过求解简化的NS方程,模拟云生成的物理过程,
这种方法耗时大,且不能有效生成预期形状的云。

基于数据驱动的云建模,在一定程度上能够克服之前建模方法的缺点,
且由于其建模过程中参考了真实数据,
能够在一定程度上反映数据信息、表达气象属性,因此具有很强的应用意义,逐渐成为研究的热点。
通过研究和分析云的相关数据(自然图像、观测数据、数值模拟数据、卫星云图等),
进一步从数据中发掘能够指导云建模的信息,
构建逼真的三维云场,同时与真实数据在形状、气象信息、规模上具有一定相关性。
相应的成果可以为天气数值预测、军事仿真、卫星气象等领域提供可视化工具和环境。

% 本论文的意义
基于时变气象数据的热带气旋的仿真技术,相比于传统的图形图像学建模方法,
由于使用的是数值模拟数据(Weather Research and Forecasting Model,简称WRF)进行建模,
模型本身就具有真实性,同时通过时序数据的矫正,使得仿真的数据气象学的意义,
符合气象学用户对数据准确性的要求;
同时又在时域内使用简单NS方程模拟云的运动过程,使得模拟速度能够达到实时应用的需求。

综述所述,热带气旋仿真技术具重要的研究意义以及广阔的应用前景,
但是现有方法建模过程难以实现及满足气象需求,同时满足逼真展现的应用。
因此,本文提出的仿真方法具有重要的意义。

% 研究目标
本论文的研究目标:针对热带气旋这一种复杂的天气系统,
设计并实现一个热带气旋的动态仿真模拟算法。 
该算法使用气象数据做初始状态,同时以长时间跨度的时变气象数据为输入,
使用基于位置的流体仿真(position based Fluids,简称PBF)方法,利用GPU加速计算云的仿真,
输出指定时间跨度的粒子数据,在保留更多细节的同时,保证仿真过度自然流畅。

在此基础上,结合基于粒子的云建模方法和
基于粒子系统的LOD(Level of Detail)层级结构的绘制方法,
形成包括建模、仿真、绘制的热带气旋动态仿真系统。

\section{论文主要研究内容}

本论文的主要研究内容包括:
提取WRF数据中的云相关建模参量建立云的粒子模型,
根据使用PBF仿真技术模拟云的运动过程,
根据WRF数据中的每一帧数据,校正PBF流体仿真产生的误差,
最后基于粒子系统的LOD层级结构,使用多散射模型的绘制方法绘制具有真实感的云。
流程图如图\ref{fig:research_contents}

本论文主要研究的内容如图\ref{fig:research_contents}所示,该图给出了整个系统的模块划分和模块之间的输入输出关系。具体内容阐述如下:
\begin{figure}[h!]
	\centering
	\includegraphics[width = 150bp]{figure/researchContemts.png}
	\caption{基于时变气象数据的热带气旋动态仿真技术研究流程图}
	\label{fig:research_contents}
\end{figure}

\begin{enumerate}

\item \textbf{基于WRF数据的云粒子建模}

粒子构建算法是将给予欧拉表征的时变气象数据通过插值、
采样等方法构建基于拉格朗日表征的差分粒子结构数据。
目的是能够更好的表示热带气旋的表观结构,
同时有利于后面的流体仿真和时序插值计算。

\item \textbf{PHF流体仿真}

流体仿真算法是本课题的一个重点研究内容,
其输入是流体的初始状态,输出为后续若干时刻的流体状态。
本题目主要的功能是进行有物理意义的长时间跨度的插值,
如果仅使用基于图形图像方面的插值算法,
会导致插值的数据不具有物理意义,不够真实。
同时对云的真实感绘制如果没有流体仿真技术的应用,
也就不具有其真实感的含义。

\item \textbf{路径计算算法}

路径计算算法的输入时气象数据的速度场数据,其输出是粒子与流线归属关系映射数据结构。
路径计算算法实现的不只是路劲的计算,同时还有实现归属映射关系的构建。
流线的生成是为了使粒子的位置完全覆盖当前时刻所有非零密度点流体的体积的同时
用尽可能少的粒子完成紧凑的计算。
构造粒子与流线归属关系的映射,为后面的误差校准提供参考依据。

\item \textbf{时序插值算法}

时序插值算法的输入时经过流体仿真输出的粒子数据和经过路径计算求得的归属关系,
输出是经过插值的粒子数据。该算法是本课题的难点。
众所周知,通过流体仿真计算得到的数据,到达下一时刻时与原数据之间是不能保持一致的,
原因就在NS方程表达的函数与我们仿真的函数是有差别的,而这种差别是不可能避免的。
因此还需要一个基于流体仿真的时序插值算法去实现流体的平滑过渡。

\item \textbf{云的绘制算法}

云的绘制算法的输入是经过平滑处理后的粒子数据,输出是仿真展现的画面。
通过以上的动态仿真,得到相应的数据,通过绘制系统绘制展现出来。
由于热带气旋属于大中尺度的天气现象,针对这种大尺度的云绘制,
一方面要实现光照的计算,同时要对大量粒子的事实绘制,云的绘制已是一个重要的研究点。

\end{enumerate}



